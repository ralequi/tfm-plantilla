% Acrónimos
%TFG
\newacronym[description={Graphic processing unit},longplural={unidades de procesamiento gráfico}]{gpu}{GPU}{unidad de procesamiento gráfico}
\newacronym{qos}{QoS}{Quality of Service}
\newacronym{dpi}{DPI}{Deep Packet Inspection}
\newacronym{dpdk}{DPDK}{Data Plane Development Kit}
\newacronym{dfa}{DFA}{Deterministic finite automaton}
\newacronym{hpet}{HPET}{High Precision Event Timer}
\newacronym{rtp}{RTP}{Real Time Protocol}
\newacronym[description={Field Programmable Gate Array},longplural={dispositivos de hardware programable}]{fpga}{FPGA}{dispositivo de hardware programable}
\newacronym[description={Network Interface Card},longplural={Interfaces de red}]{nic}{NIC}{Interfaz de red}
\newacronym[description={Núcleo lógico o Logical CORE}]{lcore}{LCORE}{logical core}

%TFM
\newacronym[description={Virtual Machine},longplural={máquinas virtuales}]{vm}{VM}{máquina virtual}
\newacronym[description={Virtual Network Function},longplural={funciones de red virtuales}]{nfv}{NFV}{función de red virtual}
\newacronym[description={Virtual Function},longplural={funciones virtuales}]{vf}{VF}{función virtual}
\newacronym[description={Internet Service Provider},longplural={proveedores de servicios de internet}]{isp}{ISP}{proveedor de servicios de Internet}
\newacronym[description={Software Defined Network},longplural={redes definidas por software}]{sdn}{SDN}{red definida por software}

% Glosario 

% TODO: Añadir aquí las definiciones del glosario
% Ejemplo de glosario
\newglossaryentry{cloud}{name={Cloud Computing},description={Se define la computación en la nube, como un conjunto de servicios que se encuentran en internet. Dichos servicios suelen estar descentralizados y repartidos en localizaciones y elementos físicos a los que usualmente no se tiene acceso}}

\newglossaryentry{cpu}{name={CPU},description={La tan conocida Unidad de procesamiento central, o CPU, se encarga de realizar la mayor parte de las tareas que realiza un ordenador de hoy en día. En las últimas arquitecturas, es también la encargada de coordinar diferentes dispositivos como la memoria o el bus PCIe.}}

\newglossaryentry{sriov}{name={SR-IOV},description={También conocida como \textit{Single Root I/O Virtualization}, SR-IOV es una tecnología de virtualización que permite a un único dispositivo PCIe, mostrarse como diversas funciones virtuales (VF). Cada una de estas VF, puede conectarse a una máquina virtual, de forma que la compartición del dispositivo sea realizada por el propio hardware. Esto, minimiza el coste computacional de la máquina anfitriona, así como de las diversas máquinas virtuales que utilizan el dispositivo}}

\newglossaryentry{passthough}{name={Passthough},description={Se denomina Passthough, al mecanismo que tienen las máquinas virtuales para apoderarse por completo de un determinado dispositivo, ``desconectandolo'' de la máquina anfitriona. De esta forma, el sistema operativo invitado, es capaz de utilizar los drivers de forma nativa con el dispositivo y en principio con perdida de rendimiento 0}}

\newglossaryentry{virtio}{name={VirtIO},description={La tecnología VirtIO, forma una parte importante de la tecnología de virtualización de KVM. El concepto de VirtIO, abarca desde los drivers de la máquina virtual, hasta los propios dispositivos virtuales generados por KVM. Un dispositivo VirtIO es, en esencia, una paravirtualización del dispositivo original}}

\newglossaryentry{mactiva}{name={monitorización activa},description={La monitorización activa consiste en la realización de una serie de medidas de una red, en donde dos o más nodos de la misma, intercambian un conjunto de paquetes. A partir de estos paquetes, se obtienen medidas como ancho de banda, jitter, latencia o cantidad de paquetes perdidos}}

\newglossaryentry{mpasiva}{name={monitorización pasiva},description={La monitorización pasiva consiste en la captura de tráfico en uno o varios puntos de la red. Analizando el tráfico capturado es posible obtener multitud de información, incluidos problemas a niveles de aplicación}}

%DOBLEGLOSARIOS:

\newglossaryentry{kvmg}{name={KVM},
    description={La tecnología KVM es una de las más populares en cuanto a virtualización se refiere, junto con su mayor competidor XEN. Esta tecnología se encuentra presente en el Kernel de Linux (en formato de módulo) y ofrece desde este nivel de ejecución la capacidad de virtualizar diferentes máquinas y dispositivos. Gracias a la tecnología actual de los procesadores (VT-X/AMD-V), esta virtualización puede llegar a ser transparente para la máquina virtualizada, además de ofrecer un aislamiento entre las diferentes máquinas virtuales y la propia máquina física}}
\newglossaryentry{kvm}{type=\acronymtype, name={KVM}, description={Kernel-based Virtual Machine}, first={máquina virtual basada en el Kernel (KVM)\glsadd{kvmg}}, see=[Glossario:]{kvmg}, firstplural={máquinas virtuales basadas en el Kernel (KVMs)\glsadd{kvmg}}}


\newglossaryentry{capexg}{name={CAPEX},
    description={El término inglés CAPEX, se refiere fundamentalmente a los gastos iniciales (o de actualización si es el caso), que son necesarios para poner en marcha un producto, o construir algo}}    
\newglossaryentry{capex}{type=\acronymtype, name={CAPEX}, description={CAPital EXpenditures}, first={gasto de capital inical (CAPEX)\glsadd{capexg}}, see=[Glossario:]{capexg}}


\newglossaryentry{opexg}{name={OPPEX},
    description={El término inglés OPPEX, se refiere fundamentalmente a los gastos que tiene un producto tras su contratación o compra}}    
\newglossaryentry{opex}{type=\acronymtype, name={OPPEX}, description={OPerational EXpenditures}, first={gasto de capital para operar (OPPEX)\glsadd{opexg}}, see=[Glossario:]{opexg}}


