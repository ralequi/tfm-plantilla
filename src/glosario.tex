% Acrónimos
%TFG
\newacronym{gpu}{GPU}{Unidad de Procesamiento Gráfico}
\newacronym{qos}{QoS}{Quality of Service}
\newacronym{dpi}{DPI}{Deep Packet Inspection}
\newacronym{dpdk}{DPDK}{Data Plane Development Kit}
\newacronym{dfa}{DFA}{Deterministic finite automaton}
\newacronym{hpet}{HPET}{High Precision Event Timer}
\newacronym{rtp}{RTP}{Real Time Protocol}
\newacronym[description={Field Programmable Gate Array}]{fpga}{FPGA}{dispositivo hardware programable}
\newacronym[description={Network Interface Card},longplural={Interfaces de red}]{nic}{NIC}{Interfaz de red}
\newacronym[description={Núcleo lógico o Logical CORE}]{lcore}{LCORE}{logical core}

%TFM
\newacronym[description={máquina virtual}]{vm}{VM}{Virtual Machine}
\newacronym[description={Funciones de red virtuales}]{nfv}{NFV}{Virtual Network Function}

% Glosario

% TODO: Añadir aquí las definiciones del glosario
% Ejemplo de glosario
%\newglossaryentry{bitstream}{name={bitstream},description={En este contexto se refiere al fichero binario que configura el Hardware de la FPGA}}


%DOBLEGLOSARIOS:

\newglossaryentry{kvmg}{name={KVM},
    description={La tecnología KVM es una de las más populares en cuanto a virtualización se refiere, junto con su mayor competidor XEN. Esta tecnología se encuentra presente en el Kernel de Linux (en formato de módulo) y ofrece desde este nivel de ejecución la capacidad de virtualizar diferentes máquinas y dispositivos. Gracias a la tecnología actual de los procesadores (VT-X/AMD-V), esta virtualización puede llegar a ser transparente para la máquina virtualizada, además de ofrecer un aislamiento entre las diferentes máquinas virtuales y la propia máquina física}}
    
\newglossaryentry{kvm}{type=\acronymtype, name={KVM}, description={Kernel-based Virtual Machine}, first={Máquina virtual basada en el Kernel (KVM)\glsadd{kvmg}}, see=[Glossario:]{kvmg}}