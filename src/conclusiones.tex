\chapter{Conclusiones y Trabajo futuro\label{sec:conclusiones}}

A lo largo de este trabajo se ha explorado el estado actual tanto de los sistemas de captura como de los sistemas de virtualización.
En base a los sistemas y equipamiento disponibles se han realizado multitud de pruebas para comparar las diferentes aplicaciones y entornos. 
Estos experimentos han desvelado los regímenes de funcionamiento de los diferentes métodos de virtualización y aplicaciones de captura de red.
Para sorpresa de muchos, la mayoría de la tecnología de virtualización actual ofrece un rendimiento equiparable a los entornos nativos de funcionamiento. De esta forma, es posible migrar los sistemas de captura a alta velocidad a los entornos virtuales. Vistos los resultados, es probable que sistemas de monitorización específicos también puedan ser migrados a entornos virtuales sin verse afectados.

En la realización de las pruebas, se han descubierto y descubierto nuevas formas de realizar una monitorización de redes virtuales que hasta ahora no habían sido propuestas o alcanzadas. Gracias a las pruebas de \gls{sriov}, se descubrieron dos posibles aplicaciones nuevas para la monitorización en entornos virtuales. Mediante las \glspl{nfv}, es posible capturar a un subconjunto de \glspl{nfv} que utilicen la misma \gls{nic}. Aunque este escenario no permita alcanzar tasas extremadamente altas, hasta donde llega mi conocimiento, no existía ninguna herramienta capaz de capturar el tráfico interno entre dos máquinas virtuales que utilizasen \gls{nfv}.
A raíz de las pruebas, también han surgido nuevas controversias. La mayoría de los capturadores o emisores de tráfico software se aferran a la filosofía del \gls{zerocopy}. No obstante, las pruebas han mostrado que utilizar de forma adecuada una filosofía \gls{onecopy} permite crear un cierto nivel de aislamiento entre el sistema de captura y el sistema de procesamiento de paquetes. Con este método, que a priori parece más lento, se puede reducir el número de estructuras de paquetes, e incluso aumentar la tasa.

Otra de las ideas surgidas a raíz de este trabajo, se trata de la monitorización distribuida. Partiendo de la suposición de disponer de múltiples máquinas virtuales que comparten \glspl{nic}, es posible instanciar nuevas máquinas virtuales que monitoricen dichas \glspl{nic} a través de \glspl{nfv}. Dichas máquinas virtuales se coordinarían con otra máquina virtual maestra y de forma distribuida podrían llegar a realizar análisis de red complejos en muy poco tiempo. Dado que los paradigmas distribuidos están tomando fuerza, parece una línea de investigación interesante por la que continuar en un futuro cercano.

A partir de este punto, se espera que los experimentos realizados ayuden a la comunidad a explotar la funcionalidad de los diferentes métodos de virtualización sin tener que enfrentarse a realizar un extenso análisis de cada uno de los elementos probados. Para lograr este objetivo, se han hecho públicos 2 dos de los tres elementos desarrollados: La librería hptl~\cite{bib:hptl} y la aplicación de solo captura sobre \gls{dpdk}~\cite{dpdkspeedometer}. Dentro del marco de publicar el trabajo desarrollado, se está trabajando actualmente en la realización de una versión \gls{onecopy} del capturador y almacenador a disco de \gls{dpdk}, con la esperanza de minimizar o eliminar las perdidas cuando el procesamiento del tráfico tiene mucha latencia.
%
Del mismo modo que se ha hecho con las herramientas construidas, se ha estado trabajando en la realización de dos artículos. El primero de ellos se encuentra aceptado en el congreso \textit{High Performance Computing and Communications} (HPCC 2015, Core B, ver anexo~\ref{sec:HPCC}). El segundo artículo se encuentra a la espera de ser enviado a una revista Q1. También se ha planteado realizar una extensión de este trabajo, añadiendo un estudio de \glspl{vf} sobre \glspl{fpga} y controladoras raid.





%A pesar de ello, aún existen ciertas combinaciones de configuraciones que no han sido exploradas y probadas.
%
%Del mismo modo, estas tecnologías se encuentran en un constante desarrollo. Cada pocos años aparece una nueva tecnología de virtualización o captura que rompe drásticamente con lo establecido. No hace tanto, se hacia desarrollo en \gls{fpga} para poder capturar y procesa tráfico a tan solo 1~\gls{gbps}, y ahora mediante unas típicas tarjetas de red y con un poco de cuidado es posible alcanzar 10 e incluso 40~\gls{gbps}. De forma similar los entornos de virtualización parecían una pérdida de tiempo y energía, y ahora la mayor parte de la computación y de los servidores del mundo se encuentran virtualizados en la nube.

%Por estos motivos, pienso que la gran mayoría de este tipo de tecnología ser encuentra aún en pañales. Un claro ejemplo es la tecnología \gls{dpdk}. Este motor de captura ha prometido dejar obsoletos al resto. No obstante, las diversas actualizaciones van alterando la API así como la eficiencia de ciertas funciones, convirtiéndola en una API poco estable para desarrollar proyectos grandes. Si comparamos \gls{dpdk} con otros driver mucho más simples como \textit{HPCAP}, vemos que aún hay margen de mejora. Una de las conclusiones más relevantes de este trabajo radica en la comparativa entre las filosofías de \gls{onecpy} y \gls{zerocpy}.


%\lsection{Trabajo futuro}